\documentclass[11pt]{ctexart}

\usepackage[S]{dryNote}
\usepackage{dryNotation}
\usepackage{dryNote}
\usepackage{tabularx}
\usepackage[toc]{multitoc}
\usepackage{gbt7714}
\usepackage{longtable}

\title{概率论与数理统计习题课-3\textsuperscript{rd}}
\author{复习复习!}
\date{2024年12月21日}

\begin{document}
\maketitle
{ 	
	\footnotesize
	\keben
	\tableofcontents
}

\addtocontents{toc}{\protect\setcounter{tocdepth}{2}}

%%%%%%%%%%%%%%%%%%%%%%%%%%%%%%%%%%%%%%%%%%%%%%%%%%%%%%%%%%%%%%%%%%%%%%%%
\section{大数定律与中心极限定律}

\subsection{基础知识}

{\keben (弱)大数定律}讨论的是, 在何种条件下, 独立同分布随机变量的均值\textbf{依概率收敛}于期望.
称随机变量序列$\{Y_n\}_{n=1}^{\infty}$依概率收敛于$Y$, 如果对任意的$\epsilon > 0$, 都有
\begin{equation*}
	\lim_{n \to \infty} \P \left\{ |Y_n - Y| \geq \epsilon \right\} = 0, 
\end{equation*}
或者, 等价地, 
\begin{equation*}
	\lim_{n \to \infty} \P \left\{ |Y_n - Y| < \epsilon \right\} = 1.  
\end{equation*}

\begin{theorem}[伯努利大数定律]
	独立同分布的\textbf{0-1随机变量}服从弱大数定律. 
	具体地说, 设$p$是每次Bernoulli试验中事件$A$发生的概率, $n_A$为$n$次试验中事件$A$发生的次数, 那么对任意的$\epsilon > 0$, 
	\begin{equation*}
		\lim_{n \to \infty} \P \left\{ \left|\frac{n_A}{n} - p \right| < \epsilon \right\} = 1.
	\end{equation*}
\end{theorem}

\begin{theorem}[辛钦大数定律]
	\textbf{期望存在}的独立同分布的随机变量服从弱大数定律.
	具体地说, 设独立同分布的随机变量$X_1, X_2, \dots$期望为$\E(X_k) = \mu$, 那么对任意的$\epsilon > 0$, 
	\begin{equation*}
		\lim_{n \to \infty} \P \left\{ \left|\frac{1}{n} \sum_{k=1}^n X_k - \mu \right| < \epsilon \right\} = 1.
	\end{equation*}
\end{theorem}


{\keben 中心极限定理}讨论的是, 在何种条件下, 独立随机变量\textbf{和的极限分布为正态分布}. 
和大数定律不同的是, 通过中心极限定理我们可以获取关于估计量分布的信息——从而提供了一种计算独立随机变量和的近似概率的简单\textbf{方法}. 
它还解释了一个引人注目的事实,即如此多的自然“种群”的经验频率呈现钟形(即正态)曲线. 

\begin{theorem}[独立同分布的中心极限定理]
	设独立同分布的随机变量$X_1, X_2, \dots$期望为$\E(X_k) = \mu$, 方差为$D(X_k) = \sigma^2 > 0$, 那么标准化的随机变量之和的极限分布服从标准正态分布: 对任意$t$, 
	\begin{equation*}
		\lim_{n \to \infty} \P \left\{ \frac{\sum_{k=1}^n X_k - n \mu}{\sqrt{n} \sigma} \leq t \right\}
		= \Phi(t). 
	\end{equation*}
\end{theorem}

下面的中心极限定理给出了二项分布的正态近似. 

\begin{theorem}[棣莫弗-拉普拉斯中心极限定理]
	设$p$是每次Bernoulli试验中事件$A$发生的概率, $n_A$为$n$次试验中事件$A$发生的次数, 那么标准化的$n_A$的极限分布服从标准正态分布: 对任意$t$, 
	\begin{equation*}
		\lim_{n \to \infty} \P \left\{ \frac{n_A - np}{\sqrt{n p (1-p)}} \leq t \right\}
		= \Phi(t). 
	\end{equation*}
\end{theorem}

\subsection{例题示范}

\begin{example}
	掷一枚质地均匀的骰子$100$次, 记$X_i$为第$i$次掷出的点数, $\overline X$为$X_1, \dots, X_{100}$的均值, 试用中心极限定理估计概率$P\{3 \leq \overline{X} \leq 4\}$.
\end{example}
\begin{solution}
	由题意可得$\E\left(X_i\right)=3.5$, $D(X_i)=\frac{35}{12}$, $\E\left(\overline X\right)=3.5$, $D(\overline X)=\frac{7}{240}$, 于是由中心极限定理, 
	\begin{align*}
		\P\{3 \leq \overline{X} \leq 4\} 
		\approx 
		\P \left\{ \frac{3 - 3.5}{\sqrt{7 / 240}} \leq \frac{\overline{X} - 3.5}{\sqrt{7 / 240}} \leq \frac{4 - 3.5}{\sqrt{7 / 240}} \right\}
		= 2 \Phi(2.9277)-1
		=0.9966. 
	\end{align*}
\end{solution}

\begin{example}
	进行独立重复试验, 每次试验中事件$A$发生的概率为$0.25$. 
	试问能以$95 \%$的把握保证1000次试验中事件$A$发生的频率与概率相差多少? 
	此时$A$发生的次数在什么范围内?
\end{example}
\begin{solution}
	记$Y_n$为1000次试验中事件$A$发生的次数, 则$Y_n \sim b(1000,0.25)$, 且 $\E\left(Y_n\right)=$ $250, D\left(Y_n\right)=187.5$.
	设事件$A$发生的频率$Y_n / 1000$与概率$0.25$的差为$k$, 我们有不等式
	\begin{align*}
		0.95 
		&\leq \P\left\{\left|\frac{Y_n}{1000}-0.25\right| \leq k\right\} 
		= \P \left\{\left|\frac{Y_n - 250}{\sqrt{187.5}}\right| \leq \frac{1000k + 0.5}{\sqrt{187.5}} \right\} \\
		&\approx 2 \Phi \left(\frac{1000k + 0.5}{\sqrt{187.5}}\right) - 1. 
	\end{align*}
	即寻求$k$, 使得
	\begin{equation*}
		\Phi \left(\frac{1000k + 0.5}{\sqrt{187.5}}\right) \geq 0.975, 
	\end{equation*}
	查表得$\frac{1000k + 0.5}{\sqrt{187.5}} \geq 1.96$, 解得$k \geq 0.027$. 
	这表明能以$95 \%$的把握保证在1000次试验中事件$A$发生的频率与概率相差不大于$0.027$.  
	或者说, 有$95 \%$的把握说, 在$1000$次试验中事件$A$发生的次数在即在$223$次到$277$次间.
\end{solution}





%%%%%%%%%%%%%%%%%%%%%%%%%%%%%%%%%%%%%%%%%%%%%%%%%%%%%%%%%%%%%%%%%%%%%%%%
\section{样本及抽样分布}

\subsection{基础知识}

数理统计的核心问题是利用样本推断总体——我们在获取数据后, 需要使用有效的方法去集中和提取数据中的有关信息, 对所研究的问题做出一定的结论, 这在统计上被称为“推断”. 
著名统计学家C.R.Rao在其著作《统计与真理——怎样运用偶然性》中概括道: 
\begin{itemize}
	\item 在终极的分析中, 一切知识都是历史;
	\item 在抽象的意义下, 一切科学都是数学;
	\item 在理性的基础上, 所有的判断都是统计学. 
\end{itemize}

\subsubsection{总体与样本}

在一个统计问题中, 我们把研究对象的全体称为\textbf{总体}, 构成总体的每个成员称为\textbf{个体}. 
而在实际问题中, 我们关心个体的数量指标值, 于是总体就是一堆数——有大有小、有的出现机会大, 有的出现机会小. 
因此用一个概率分布去描述和归纳总体是恰当的, 从这个意义看, 总体就是一个随机变量$X$. 

对总体的观察称为\textbf{抽样}: 我们从总体$X$中进行$n$次独立地、重复地观察, 得到$n$个结果$X_1, \dots, X_n$, 称为来自总体$X$的、容量为$n$的\textbf{简单随机样本}, 简称\textbf{样本}.
于是$X_1, \dots, X_n$\textbf{独立}且与总体$X$有\textbf{相同分布}$F$, 记做$X_1, \dots, X_n$ i.i.d. $\sim F$. 

样本具有\textbf{二重性}: 一方面, 由于样本是从总体中随机抽取的, 抽取前无法预知它们的数值, 因此样本是随机变量, 用大写字母$X_1, \dots, X_n$表示; 另一方面, 样本在抽取以后经观测就有确定的观测值, 因此样本又是一组数值, 此时用小写字母$x_1, \dots, x_n$表示是恰当的.

\subsubsection{统计量}

由于样本本身是一些杂乱无章的数字, 我们需要对数字加工处理, 计算出一些有用的量. 
\begin{definition}[统计量]
	由样本算出来的量称为统计量, 或者说, 统计量是样本的函数. (注意: 统计量中不能含有未知参数.)
\end{definition}
下面列出一些常见的统计量. 
\begin{itemize}
	\item \textbf{样本均值} $\displaystyle \overline{X} = \frac{1}{n} \sum_{i=1}^n X_i$; 
	\item \textbf{样本方差} $\displaystyle S^2 = \frac{1}{n-1} \sum_{i=1}^n (X_i - \overline{X})^2 = \frac{1}{n-1} \left( \sum_{i=1}^n X_i^2 - n  \overline{X}^2 \right)$; 
	\item \textbf{样本标准差} $\displaystyle S = \sqrt{S^2}$; 
	\item \textbf{样本$k$阶(原点)矩} $\displaystyle A_k = \frac{1}{n} \sum_{i=1}^n X_n^k$, $k = 1, 2, \dots$; 
	\item \textbf{样本$k$阶中心矩} $\displaystyle B_k = \frac{1}{n} \sum_{i=1}^n (X_n - \overline{X})^k$, $k = 1, 2, \dots$. 
\end{itemize} 

\begin{proposition}\label{prop:sample}
	设$X_1, \dots, X_n$为来自总体$X$的样本, $\E(X) = \mu$, $D(X) = \sigma^2$, 那么
	\begin{equation*}
		\E(\overline{X}) = \mu,\;
		D(\overline{X}) = \frac{\sigma^2}{n},\;
		\E(S^2) = \sigma^2. 
	\end{equation*}
\end{proposition}
\begin{proof}
	我们只证明样本方差的期望是$\sigma^2$. 
	注意到$X_k$两两独立意味着
	\begin{equation*}
		\E(\overline{X}^2) 
		= \frac{1}{n} \sum_{i=1}^n \E(X_i \cdot \overline{X}) 
		= \E(X_k \cdot \overline{X}), 
	\end{equation*}
	其中
	\begin{equation*}
		\E(X_k \cdot \overline{X})
		= \frac{1}{n} \sum_{i=1}^n \E(X_k \cdot X_i)
		= \frac{n-1}{n} \E(X)^2 + \frac{1}{n} \E(X^2). 
	\end{equation*}
	于是
	\begin{align*}
		\E(S^2)
		&= \frac{1}{n-1} \sum_{i=1}^n \E (X_i - \overline{X})^2
		= \frac{n}{n-1} \left[ \E(X_k^2) - 2 \E(X_k \cdot \overline{X}) + \E(\overline{X}^2) \right] \\
		&= \frac{n}{n-1} \left[ \E(X^2) - \frac{n-1}{n} \E(X)^2 - \frac{1}{n} \E(X^2) \right] 
		= \E(X^2) - \E(X)^2 \\
		&= D(X) = \sigma^2. 
	\end{align*}
\end{proof}
\begin{remark}
	在统计学中, \textbf{自由度}是指当以样本的统计量来估计总体的参数时,样本中独立或能自由变化的数据的个数——这里的“度”指的是“维度”. 
	若样本$X_1, X_2, X_3$的均值$\overline{X}$是事先指定的, 那么此时真正取值“自由”的随机变量只有$2$个: 
	例如, 当$X_1, X_2$已知时, $X_3$不得不取$3 \overline{X} - X_1 - X_2$. 
	而在计算$n$个样本的方差时, 样本均值已经给定, 此时自由度是$n-1$. 
\end{remark}


\subsubsection{经验分布函数}

若我们有总体$X$的$n$个观察值$x_1, \dots, x_n$, 我们可以写出分布的经验分布函数
\begin{equation*}
	F_n(x) = \frac{\# \{x_i \colon x_i \leq x\}}{n},  
\end{equation*}
即样本观察值小于等于$x$的频率. 
可以看到这样的函数满足分布函数的非负、单增、右连续三个条件. 

Glivenko–Cantelli 定理指出, 样本容量极大时,样本的经验分布趋近于总体分布(a.e.), 这被称为推断统计学的基石. 

\subsubsection{三大抽样分布}

作为样本的函数, 统计量也是随机变量, 它的分布称为\textbf{抽样分布}. 
以\textbf{标准正态变量}为基石而构造的三个著名统计量在实际中有广泛的应用, 它们被称作数理统计中的\textbf{三大抽样分布}. 

\begin{definition}[卡方分布]
	设$X_1, \dots, X_n$ i.i.d. $\sim N(0,1)$, 则称随机变量
	\begin{equation*}
		\chi^2 = \sum_{i = 1}^n X_n^2
	\end{equation*}
	服从\textbf{自由度为$n$的$\chi^2$分布}, 记做$\chi \sim \chi^2(n)$. 
\end{definition}
由定义, 不难看出卡方分布关于\textbf{自由度}具有\textbf{可加性}, 即若$\chi_1$与$\chi_2$相互独立, $\chi_1 \sim \chi^2(n),\; \chi_2 \sim \chi^2(m)$, 那么
\begin{equation*}
	\chi_1 + \chi_2 \sim \chi^2(n+m). 
\end{equation*}
此外我们还需熟记卡方分布的期望与方差: 若$\chi \sim \chi^2(n)$, 那么
\begin{equation*}
	\E(\chi) = n, \; D(\chi) = 2n. 
\end{equation*}

\begin{definition}[t分布]
	设$X \sim N(0,1)$, $Y \sim \chi^2(n)$且$X$与$Y$相互独立, 则称随机变量
	\begin{equation*}
		t = \frac{X}{\sqrt{Y / n}}
	\end{equation*}
	服从\textbf{自由度为$n$的t分布}, 记做$t \sim t(n)$.
\end{definition}
注意$t$分布的密度函数和标准正态曲线类似, 呈现“中间高, 两边低, 左右对称”, 但是两侧的尾部概率比$N(0,1)$大. 
于是$t$分布的上分位数满足$t_{\alpha}(n) + t_{1 - \alpha}(n) = 0$(互为相反数). 

\begin{definition}[F分布]
	设$U \sim \chi^2(m)$, $V \sim \chi^2(n)$且$U,V$相互独立, 则称随机变量
	\begin{equation*}
		F = \frac{U / m}{V / n}
	\end{equation*}
	服从\textbf{自由度为$m,n$的F分布}, 记做$F \sim F(m, n)$.
\end{definition}
由定义不难看出$t^2(n) = F(1,n)$. 
此外$\frac{1}{F} \sim F(n, m)$, 于是$F$分布的上分位数满足$F_{\alpha}(m,n) \cdot F_{1-\alpha}(n,m) = 1$(互为倒数): 
\begin{equation*}
	1 - \alpha 
	= \P(F \leq F_{\alpha}(m,n)) 
	= \P \left( \frac{1}{F} \geq \frac{1}{F_{\alpha}}(m,n) \right). 
\end{equation*}  

%三大抽样分布的上分位数可以通过查表得出. 

在命题\ref{prop:sample}的基础上, 若总体$X$服从正态分布, 我们可以构造出一些常用统计量, 在假设检验中有着很广泛的应用. 


\begin{theorem}[单正态总体的抽样分布]
	设样本$X_1, X_2, \dots X_n$ i.i.d. $\sim N(\mu, \sigma^2)$, $\overline{X}$, $S^2$分别为样本均值与方差, 那么
	\begin{enumerate}
		\item $\overline{X} \sim N(\mu, \sigma^2 / n)$, 于是$U = \frac{\overline X - \mu}{\sigma / \sqrt{n}} \sim N(0,1)$; 
		\item $\frac{(n-1) S^2}{\sigma^2} \sim \chi^2(n-1)$; {\color{cyan} 最关键的一点!请务必记住.}
		\item $\overline{X}$与$S^2$独立, 于是
			\begin{equation*}
				\frac{\overline{X} - \mu}{S / \sqrt{n}} \sim t(n-1). 
			\end{equation*}
	\end{enumerate}	
\end{theorem}
我们可以看到, 最终构造的统计量$\frac{\overline{X} - \mu}{S / \sqrt{n}}$\textbf{不含总体方差这一未知参数}, 并且分布信息可以通过查表得到——从而可以在$\sigma$未知的情况下, 给出$\mu$的区间估计. 

有的时候, 我们需要比较两个正态总体的参数, 下面的统计量允许我们在参数$\mu_1, \mu_2$未知的情况下, 对$\sigma_1 / \sigma_2$做出推断. 
\begin{theorem}[双正态总体的抽样分布]
	设样本$X_1, X_2, \dots X_m$ i.i.d. $\sim N(\mu_1, \sigma_1^2)$, $Y_1, Y_2, \dots Y_n$ i.i.d. $\sim N(\mu_2, \sigma_2^2)$, 且两样本之间相互独立.  
	$\overline X, \overline Y, S_1^2, S_2^2$分别为两个样本的均值与方差, 那么
	\begin{equation*}
		\frac{S_1^2 / \sigma_1^2}{S_2^2 / \sigma_2^2} \sim F(m-1, n-1). 
	\end{equation*}
\end{theorem}
\begin{proof}
	在上一定理的基础上, 我们有$\frac{(m-1) S_1^2}{\sigma_1^2} \sim \chi^2(m-1)$, $\frac{(n-1) S_1^2}{\sigma_1^2} \sim \chi^2(n-1)$, 于是由F分布的定义可知定理成立. 
\end{proof}

\subsection{例题示范}

\begin{example}
	设$x_1, \dots x_{16}$为来自$N(\mu, \sigma^2)$的样本, 经计算$\bar x = 9$, $s^2 = 5.32$, 试求$\P\{|\bar x - \mu| < 0.6\}$. 
\end{example}
\begin{solution}
	对于单正态总体, 我们考虑构造服从$t$分布的统计量. 
	由于$\bar x - \mu \sim N(0, \sigma^2 / n)$, $\frac{(n-1) s^2}{\sigma^2} \sim \chi^2(n-1)$, 那么
	\begin{align*}
		t 
		:= \frac{\frac{\bar x - \mu}{\sigma / \sqrt{n}}}{\sqrt{\frac{(n-1) s^2}{\sigma^2} / (n-1)}}
		= \frac{\sqrt{n}(\bar x - \mu)}{s} 
		= \frac{4}{\sqrt{5.32}} (\bar x - \mu)
		\sim t(15). 
	\end{align*}
	于是
	\begin{align*}
		\P\{|\bar x - \mu| < 0.6\}
		&= \P\left\{|\bar x - \mu| < 0.6\right\}
		= \P\left\{|t| < \frac{4 \times 0.6}{\sqrt{5.32}} \right\} \\
		&
%		= 2 t_{1.04}(15) - 1
		= 0.6854. 
	\end{align*}
\end{solution}

\begin{example}
	若随机变量$X \sim F(n,n)$, 证明$\P\{X<1\} = 0.5$. 
\end{example}
\begin{proof}
	由F分布的定义有$Y := 1/X \sim F(n,n)$与$X$同分布, 于是
	\begin{equation*}
		\P\{X < 1\}
		= \P\{Y < 1\}
		= \P\{1/X < 1\}
		= \P\{X > 1\}. 
	\end{equation*}
	另一方面, 由于$X$是连续型随机变量, 我们有$\P\{X < 1\} + \P\{X > 1\} = 1$, 于是$\P\{X < 1\} = 0.5$. 
\end{proof}

\begin{example}
	设$x_1, \dots, x_{15}$是总体$N(0, \sigma^2)$的一个样本, 求
	\begin{equation*}
		\frac{x_1^2 + \dots + x_{10}^2}{2(x_{11}^2 + \dots + x_{15}^2)}
	\end{equation*}
	的分布. 
\end{example}
\begin{solution}
	由于$x_1, \dots x_n$ i.i.d. $N(0,1)$, 那么
	\begin{align*}
		\frac{1}{\sigma^2}(x_1^2 + \dots + x_{10}^2) \sim \chi^2(10), \; 
		\frac{1}{\sigma^2}(x_{11}^2 + \dots + x_{15}^2) \sim \chi^2(5), 
	\end{align*}
	于是
	\begin{equation*}
		\frac{x_1^2 + \dots + x_{10}^2}{2(x_{11}^2 + \dots + x_{15}^2)}
		= \frac{\frac{1}{\sigma^2}(x_1^2 + \dots + x_{10}^2) / 10}{\frac{1}{\sigma^2}(2(x_{11}^2 + \dots + x_{15}^2) / 5}
		\sim F(10, 5). 
	\end{equation*}
\end{solution}


%%%%%%%%%%%%%%%%%%%%%%%%%%%%%%%%%%%%%%%%%%%%%%%%%%%%%%%%%%%%%%%%%%%%%%%%
\section{参数估计}

\subsection{基础知识}

\subsubsection{点估计}

对于未知参数$\theta$, 我们选用统计量$\hat{\theta} = \hat{\theta}(X_1, X_2, \cdots, X_n)$的取值作为$\theta$的一个估计量, $\hat{\theta}(x_1, x_2, \cdots, x_n)$作为估计值. 
常用的点估计有\textbf{矩估计}和\textbf{最大似然估计}. 

\noindent{\large \keben 矩估计}

矩估计的基本想法是“替换原理”: 用相应的样本矩(的函数)\textbf{替换}总体矩(的函数), 并且尽量选取低阶矩.
矩估计的理论依据是辛钦大数定律
\begin{equation*}
	A_k = \frac{1}{n} \sum_{i=1}^n X_i^k \stackrel{\P}{\to} \E(X^k). 
\end{equation*}

例如若总体$X$的期望$\mu$与方差$\sigma^2$未知, 我们通常选用一阶样本原点矩(即样本均值)和二阶样本中心距分别作为二者的矩估计: 
\begin{equation*}
	\hat{\mu} = \overline X = \frac{1}{n} \sum_{i=1}^n X_i, \quad
	\hat{\sigma}^2 = \frac{1}{n} \sum_{i=1}^n(X_i - \overline X)^2. 
\end{equation*}


\noindent{\large \keben 最大似然估计}

最大似然估计法是求点估计用得最多的方法, 它的基本想法是, 未知参数$\theta$的估计应该是使得当前结果出现可能性最大的那个$\theta$. 

设$X_1, \dots, X_n$为来自总体$X$的样本, $X$的分布已知而参数$\theta$未知. 
最大似然估计的方法步骤为: 
\begin{enumerate}
	\item 求出当前事件出现的概率/密度, 它是样本和未知参数的函数$L(X_1, \dots, X_n; \theta)$, 我们称之为\textbf{似然函数}; 
	\item 寻求使得似然函数达到最大的点$\hat \theta$作为$\theta$的估计
		\begin{equation*}
			\hat \theta = \argmax_{\theta \in \Theta} L(X_1, \dots, X_n; \theta). 
		\end{equation*}
		其中$\Theta$为参数$\theta$的取值范围, 例如方差的参数空间是$\R_{\geq 0}$. 
\end{enumerate}

\noindent
若$X$为离散型随机变量, 它的分布律为$\P\{X = x\} = p(x; \theta)$, 由样本的独立性, 
\begin{equation*}
	L(x_1, \dots, x_n; \theta) 
	:= \P\{X_1 = x_1, \cdots, X_n = x_n\}
	= \prod_{i=1}^n p(x_i; \theta). 
\end{equation*}
若$X$为连续型随机变量, 它的密度为$f(x;\theta)$, 此时似然函数为
\begin{equation*}
	L(x_1, \dots, x_n; \theta) 
	:= \prod_{i=1}^n f(x_i; \theta).
\end{equation*}

\begin{remark}
	大多情况下, 似然函数$L$关于$\theta$可微, 于是极值点可由似然方程
	\begin{equation*}
		\frac{\dd}{\dd \theta} L(X_1, \dots, X_n; \theta) = 0
	\end{equation*}
	解出驻点, 再得到最大值点.
	特别当驻点唯一时, 最大值点只能是此驻点. 
	进一步地, 由于似然函数$L$和它的对数函数$\ln L$同时取得最值, 实际计算中我们也常用对数似然方程
	\begin{equation*}
		\frac{\dd}{\dd \theta} \ln L(X_1, \dots, X_n; \theta) = 0
	\end{equation*}
	解出(对数可以把乘积转换为求和的形式, 便于求导, 特别对正态分布、指数分布这种密度包含指数函数的情况, 能够极大的方便我们的计算). 
\end{remark}

若有多个未知参数$\theta_1, \dots, \theta_m$, 则从似然方程组$\frac{\partial}{\partial \theta_i} L$, $i = 1, \dots, m$或者对数似然方程组解出驻点, 再判断驻点是否为最大值点. 

\noindent{\large \keben 估计量的评选标准}

由于不同方法可能得到不同的点估计, 我们需要对不同的估计量做出评价. 
常用的标准有\textbf{无偏性}、\textbf{有效性}及\textbf{相合性}等, 这需要点估计量分别满足如下要求: 
\begin{itemize}
	\item \textbf{无偏性}: 称$\hat \theta$是$\theta$的无偏估计量, 如果$\E(\hat \theta) = \theta$; 
	\item \textbf{有效性}: 称无偏估计量$\hat \theta_1$比无偏估计量$\hat \theta_2$更有效, 如果对任意$\theta \in \Theta$, 都有$D(\hat \theta_1) \leq D(\hat \theta_2)$且存在某个$\theta$使得不等号严格成立; 
	\item \textbf{相合性}: 称$\hat \theta = \hat \theta(X_1, \dots, X_n)$为参数$\theta$的相合估计量, 如果$\hat \theta \stackrel{\P}{\to} \theta$. 
\end{itemize}
\begin{remark}
	由辛钦大数定律, 样本的二阶中心矩是方差的相合估计, 但不是无偏估计——由命题\ref{prop:sample}, 样本方差才是总体方差的无偏估计, 此外样本方差也是总体方差的相合估计. 
\end{remark}

\subsubsection{区间估计}

%点估计只是给出了未知参数的一个近似值, 并不能很好地反映误差. 
由于时间缘故, 本节未能来得及准备, 私密马赛. 

\subsection{例题示范}

\begin{example}
	设$x_1, x_2, x_3$是来自总体的样本, 是证明下列统计量都是总体均值$\mu$的无偏估计, 并比较方差存在时的有效性. 
	\begin{enumerate}[label = (\arabic*)]
		\item $\hat \mu_1 = \frac12 x_1 + \frac13 x_2 + \frac16 x_3$; 
		\item $\hat \mu_2 = \frac13 x_1 + \frac13 x_2 + \frac13 x_3$;
		\item $\hat \mu_3 = \frac16 x_1 + \frac16 x_2 + \frac23 x_3$. 
	\end{enumerate}
\end{example}
\begin{solution}
\begin{align*}
& \E\left(\hat{\mu}_1\right)=\frac{1}{2} \E\left(x_1\right)+\frac{1}{3} \E\left(x_2\right)+\frac{1}{6} \E\left(x_3\right)=\frac{1}{2} \mu+\frac{1}{3} \mu+\frac{1}{6} \mu=\mu, \\
& \E\left(\hat{\mu}_2\right)=\frac{1}{3} \E\left(x_1\right)+\frac{1}{3} \E\left(x_2\right)+\frac{1}{3} \E\left(x_3\right)=\frac{1}{3} \mu+\frac{1}{3} \mu+\frac{1}{3} \mu=\mu, \\
& \E\left(\hat{\mu}_3\right)=\frac{1}{6} \E\left(x_1\right)+\frac{1}{6} \E\left(x_2\right)+\frac{2}{3} \E\left(x_3\right)=\frac{1}{6} \mu+\frac{1}{6} \mu+\frac{2}{3} \mu=\mu.
\end{align*}
于是它们都是$\mu$的无偏估计. 
不妨设总体方差为$\sigma^2$, 那么
\begin{align*}
&D\left(\hat{\mu}_1\right)=\frac{1}{4}D\left(x_1\right)+\frac{1}{9}D\left(x_2\right)+\frac{1}{36}D\left(x_1\right)=\frac{1}{4} \sigma^2+\frac{1}{9} \sigma^2+\frac{1}{36} \sigma^2=\frac{7}{18} \sigma^2, \\
&D\left(\hat{\mu}_3\right)=\frac{1}{9}D\left(x_1\right)+\frac{1}{9}D\left(x_2\right)+\frac{1}{9}D\left(x_3\right)=\frac{1}{9} \sigma^2+\frac{1}{9} \sigma^2+\frac{1}{9} \sigma^2=\frac{1}{3} \sigma^2, \\
&D\left(\hat{\mu}_3\right)=\frac{1}{36}D\left(x_1\right)+\frac{1}{36}D\left(x_2\right)+\frac{4}{9}D\left(x_3\right)=\frac{1}{36} \sigma^2+\frac{1}{36} \sigma^2+\frac{4}{9} \sigma^2=\frac{1}{2} \sigma^2. 	
\end{align*}
不难看出$D(\hat \mu_2) < D(\hat \mu_1) < D(\hat \mu_3)$, 于是$\hat \mu_2$最有效. 
\end{solution}

\begin{example}
	设总体密度函数如下, $x_1, \dots, x_n$为来自总体的样本, 试求未知参数的矩估计: 
	\begin{enumerate}[label = (\arabic*)]
		\item $p(x;\theta) = (\theta+1) x^{\theta}$, $0<x<1$, $\theta > 0$; 
		\item $p(x; \theta, \mu) = \frac{1}{\theta} e^{- \frac{x - \mu}{\theta}}$, $x > \mu$, $\theta > 0$. 
	\end{enumerate}
\end{example}
\begin{solution}
	\begin{enumerate}[label = (\arabic*)]
		\item 由$\E(X) = \int_0^1 x \cdot p(x;\theta) \dd x = \frac{\theta+1}{\theta+2}$, 于是$\theta = \frac{1-2 \E(X)}{\E(X) - 1}$, 从而由替换原理, 矩估计$\hat \theta = \frac{1-2 \bar x}{\bar x - 1}$; 
		\item 总体的期望和方差分别为
			\begin{align*}
				\E(X) 
				&= \int_{\mu}^{\infty} x \cdot p(x;\theta) \dd x
				= \int_0^{\infty} \frac{t}{\theta} e^{-\frac{t + \mu}{\theta}} \dd t 
				= \theta + \mu, \\
				\E(X^2)
				&= \int_{\mu}^{\infty} x^2 \cdot p(x;\theta) \dd x
				= \int_0^{\infty} \frac{(t+\mu)^2}{\theta} e^{-\frac{t}{\theta}} \dd t
				= 2 \theta^2 + 2 \mu \theta + \mu^2, \\
				D(X)
				&= \E(X^2) - \E(X)^2
				= \theta^2. 
			\end{align*}
			于是$\theta = \sqrt{D(X)}$, $\mu = \E(X) - \sqrt{D(X)}$, 从而由替换原理, 矩估计$\hat \theta = \sqrt{\frac{1}{n} \sum_{i=1}^n (x_i - \bar x)^2}$, $\hat \mu = \bar x -  \sqrt{\frac{1}{n} \sum_{i=1}^n (x_i - \bar x)^2}$. 
	\end{enumerate}
\end{solution}


\begin{example}[均匀分布的矩估计]
	设样本$X_1, \dots, X_n$来自均匀分布$U(a, b)$, 给出未知参数$a, b$的矩估计. 
\end{example}
\begin{solution}
	由于$\E(X) = \frac{a+b}{2}$, $D(X) = \frac{(b-a)^2}{12}$, 不难得出
	\begin{equation*}
		a = \E(X) - \sqrt{3 D(X)},\;
		b = \E(X) + \sqrt{3 D(X)}. 
	\end{equation*}
	我们用一阶样本原点矩和二阶样本中心矩分别替换$\E(X)$和$D(X)$即可:
	\begin{equation*}
		\hat a = \overline X - \sqrt{\frac{3}{n} \sum_{i=1}^n (X_i - \overline X)^2},\; 
		\hat b = \overline X + \sqrt{\frac{3}{n} \sum_{i=1}^n (X_i - \overline X)^2}. 
	\end{equation*}
\end{solution}


\begin{example}[均匀分布的最大似然估计]
	{\color{blue} 参照教材例6}. 
\end{example}

\begin{example}
	设总体概率密度如下, $x_1, \dots, x_n$为来自总体的样本, 试求未知参数的最大似然估计:
	\begin{enumerate}[label = (\arabic*)]
		\item $p(x;\theta) = \sqrt{\theta} x^{\sqrt{\theta} - 1}$, $0<x<1$, $\theta > 0$; 
		\item $p(x; \theta, c) = \frac{1}{\theta} e^{- \frac{x - c}{\theta}}$, $x > c$, $\theta > 0$. 
	\end{enumerate}
\end{example}
\begin{solution}
	\begin{enumerate}[label = (\arabic*)]
		\item 似然函数为$L(\theta) = \theta^{\frac{n}{2}} (x_1 x_2 \cdots x_n)^{\sqrt{\theta} - 1}$, 于是对数似然函数为
			\begin{equation*}
				\ln L(\theta) = \frac{n}{2} \ln \theta + (\sqrt{\theta} - 1) \sum_{i=1}^n \ln x_i.
			\end{equation*}
			于是似然方程为
			\begin{equation*}
				\frac{\dd}{\dd \theta} \ln L(\theta) = \frac{n}{2 \theta} + \frac{\sum_{i=1}^n \ln x_i}{2 \sqrt{\theta}} = 0, 
			\end{equation*}
			解之得唯一驻点$\displaystyle \hat \theta = \left( \frac{1}{n} \sum_{i=1}^n \ln x_i \right)$. 
		\item 记$x_{(1)} = \min\{x_1, \dots, x_n\}$, 此时似然函数为
			\begin{equation*}
				L(\theta) = \theta^{-n} \exp\left\{ -\frac{1}{\theta} \sum_{i=1}^n (x_i - c) \right\},\; c < x_{(1)}, 
			\end{equation*}
			注意到对数似然函数
			\begin{equation*}
				\ln L(\theta) = - n \ln \theta - \frac{1}{\theta} \sum_{i=1}^n (x_i - c),\; c < x_{(1)}
			\end{equation*}
			关于$c$单调增, 要使似然函数达到最大, $c$应在限制范围$c < x_{(1)}$内尽可能取大, 于是最大似然估计$\hat c = x_{(1)}$. 
			另一方面, 我们令$\ln L(\theta, \hat c)$关于$\theta$的导数为$0$解出似然方程得
			\begin{equation*}
				\hat \theta = \frac{\sum_{i=1}^n (x_i - \hat c)}{n} = \bar x - x_{(1)}.
			\end{equation*}
	\end{enumerate}
\end{solution}






\end{document}